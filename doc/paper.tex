\documentclass{llncs}

\usepackage{mylncs}
\usepackage{multirow}
 
\title{Competitiveness in Bridge Bidding \\ An Historical Analysis}
\titlerunning{Competitiveness in bridge bidding}
\author{Guus Schreiber}
\institute{VU University Amsterdam, Computer Science, 
De Boelelaan 1085, 1081 HV Amsterdam, The Netherlands. 
E-mail: \instance{guus.schreiber@vu.nl}}
\authorrunning{Guus Schreiber}   
\tocauthor{Guus Schreiber, VU University Amsterdam}

\begin{document}

\maketitle

\begin{abstract}
In this article we analyse historical data to find out whether bridge bidding has changed during the past decades. As data set we use a collection of 26,00 records of deals played in Bermuda Bowl competitions from 1955-2013.      
\end{abstract}

\section{Introduction}

Expert bridge players regularly discuss the fact bridge bidding has
changed over years. In particular, it is said that bridge bidding has
become much more competitive. This statement is almost certainly true,
but there exists little to no statistical evidence
to support that claim. The objectie of this article is to  analyse
historical records of expert bridge bidding in order to provide such
evidence. One example analysis we want to do is to see wether and to
what extent the
frequency of opening a balanced hand with 11-12 HCP increased. In
order to do this we need to analyse for each deal both the hands and the auction. 
This means we require some limited form of reasoning on top of
database querying.  As data we use in this study a set of some
20,000  hand records played during Bermuda Bowl championships in te
period 1955-2013.  

Taking a wider perspective, we see this study as one possible step
towards  a more evidence-based approach to the theory of bridge
bidding. Through channels such as 
Bridge Base Online\footnote{\url{http://www.bridgebase.com}} more and
more hand records become available for analysis. This provides us with
data that can be used to analyse the effectiveness of particular bidding
agreements. An example would be a comparison of the Multi opening and
the natural Weak Two. 

In the next section we discuss the overall approach. \secref{data} data used in this
study. \secref{data} describes the data and the data
pre-processing. In \secref{analysis} we use the data to get
statistical data about a number of questions that allows us to get more
insight into whether and how bidding has changed. 
All data and software of this study can be found in GitHub%
\footnote{\url{https://github.com/guusschreiber/bidstat}}.


\section{Approach}
\label{sec:approach}

We used hand records encoded with in the Portable Bridge Notation
version 2.1 \cite{pbn21}. This plain-text notation is derived from a popular
notation for chess (PGN) and adapted to bridge by members of the
Internet newsgroup \instance{rec.games.bridge}. Several people have
published sets of hand records in PBN format. Another popular format
is the LIN (.lin) format used by BBO. Converters between these PBN and
LIN exists.   

Each hand record contains metadata about the event (e.g., time, place,
tournament), table data (e.g., player names) and deal data (e.g.,
vulnerability, hands, auction, play).  In this study we are mainly
focus on data about hands and auction. We process the PBN data in
four steps:

\begin{enumerate}
\item We first transform the PBN
text with an information-preserving Perl parser into a Prolog
representation. We opt for Prolog because logic programming is
excellently suited for the type of reasoning we need to do with the
data. 
\item In the second step we derive for each deal Prolog facts about
the hands and the auction, for example that the player in second
position declarer has an \type{unbalanced}
hand with a \type{1246} distribution and \type{8 HCP}, and has made a
\type{jump overcall} at the \type{3 level}. 
\item Subsequently, we create data tables that contain for every deal
the year and one or more values a values for a particular bidding
feature, such as whether it is a \type{contested auction}.   
\item In the final step we process the data tables with statistical
software (the R language). 
\end{enumerate}

In the next section we discuss the data processing in step 2 in some
detail. In \secref{analysis} we discuss the results of a number of
analyses.  

\section{Data}
\label{sec:data}

In this study we analysed 19,660 hand records\footnote{%
The PBN files contained 19,724 records; 64 records were put aside
because no (correct) hand or auction information was available in the record.}
from 23 Bermuda Bowls
over a period of 60 years (see \tblref{data}).  The number of deals per
tournament varies; recent tournaments typically have more records.  We
selected the Bermuda Bowl as data set because it
is the only tournament for which data is available over a long period
of time.   

\begin{table}
\centering\footnotesize
\caption{Hand records used in this study. Columns indicate decades;
  rows a particular year within  a decade. Hand records with
  incomplete or erroneous hand or auction data are left out.}
\label{tbl:data}
\begin{tabular}{|c|r|r|r|r|r|r|r|r|r|r|r|}
\hline
\bf\ Decade/Year \ &
\bf \ \ \ 0 \ &\bf \ \ \ 1 \ &\bf \ \ \ 2 \ &\bf \ \ \ 3 \ &\bf \ \ \ 4 \ &
\bf \ \ \ 5 \ &\bf \ \ \ 6 \ &\bf \ \ \ 7 \ &\bf \ \ \ 8 \ &\bf \ \ \ 9 \ &
\bf \ Total \ \\ \hline\hline
50's &  &  &  &  &  & 448 &  & 447 &  & 312 & 1207 \\ \hline
60's &  &  & 576 &  &  &  &  & 256 &  &  & 832 \\ \hline
70's &  &  &  & 256 & 192 & 191 &  & 192 &  & 192 & 1023 \\ \hline
80's &  & 192 &  & 352 &  &  &  & 350 &  &  & 894 \\ \hline
90's &  & 169 &  &  &  & 319 &  & 1211 &  &  & 1699 \\ \hline
00's & 702 & 317 &  & 1593 &  & 1994 &  & 1527 &  & 3114 & 9247 \\ \hline
10's &  & 3512 &  & 1246 &  &  &  &  &  &  & 4758 \\ \hline\hline
Total &  &  &  &  &  &  &  &  &  &  & 19660 \\ \hline
\hline
\end{tabular}
\end{table}

From each record the following data is used: 
\begin{itemize}
\item year in which it was played
\item the four hands (list with dealer first)
\item vulnerability
\item auction, including final contract and declarer
\item result (number of tricks; 0 if passed out)
\end{itemize}

The data is represented in a computation-friendly way. For example,
the auction is represented as a Prolog list of numbers. Her is one
sample auction\footnote{%
Auction on board 1 of the 1955 Bermuda Bowl between USA and Great
Britain. NS: Mathe-Rosen; EW: Reese-Shapiro.}:

\begin{code}
    [12,13,7,0,0,21,22,31,32,0,0,0]
\end{code}

Each number encodes a bid: 12 stands for
1$\diamondsuit$, 13 for 1$\heartsuit$, 21 for 2$\clubsuit$, etc. The numbers
0 and 7 encode a pass and a double, respectively. 

For each record we derived three types of additional facts: 
\begin{itemize}
\item Hand features that can be derived from the 13 cards
of one player, e.g. high-card points (HCP), distribution,
balanced/semi-balanced/unbalanced, 1/2/3-suited.
\item Features of a bid independent of the auction,
e.g. level, major/minor/NT, 
\item Features of a bid dependent on the auction, e.g. opening,
overcall (direct, ``live''), single/double/... jump. 
\end{itemize}
 
These features form the basis for generating the data tables used for
analysis. 


\section{Analysis}
\label{sec:analysis}

\subsection{Preliminaries}

As stated, our objective is to analyse whether competitive bidding
style  has changed over time.  Before diving into this, we need to consider
whether the fact that in earlier days were not computer-dealt
influences the data set. We have not studied this feature in every detail,
but \tblref{balanced-first-hand} gives some indication. In this table
we see the frequency of a having as dealer a balanced hand. This table
suggest there is no strong effect of dealing by hand (or at least at
the Bermuda Bowl manual dealing was methodical). Nevertheless, when we
formulate concrete research questions we will take care that, whenever
possible, this factor is ruled out.     

\begin{table}
\caption{Frequency of balanced first hand}
\label{tbl:balanced-first-hand}
\centering\footnotesize
\begin{tabular}{|c|r|r|r|}
\hline
\bf \ Decade \ & \bf \ Balanced \ & \bf \ Total \  & \bf \ \% Balanced \ \\ \hline 
50-59 & 600 & 1207 & 0.50 \\
60-69 & 542 & 832 & 0.65 \\
70-79 & 492 & 1023 & 0.48 \\
80-89 & 502 & 894 & 0.56 \\
90-99 & 902 & 1699 & 0.53 \\
00-09 & 4740 & 9247 & 0.51 \\
10-13 & 2565 & 4758 & 0.54 \\ \hline
Total & 10343 & 19660 & 0.53 \\
\hline
\end{tabular}
\end{table}

For this study we selected four concrete questions to get insight into the
overall issue of competitiveness: 
\begin{enumerate}
\item How frequent is a balanced hand with 11-12 HCP opened in first
  and second position?
\item If the dealer holds a six-card suit with 0-9 HCP, is the hand
  opened and at which level? 
\item What is the relationship between suit length and level in
  preemptive openings bids and overcalls? 
\item Whats is the frequency of contested auction'?   
\end{enumerate}

The questions cover by no means the full spectrum of competitiveness,
but will hopefully give us some insight into the issue.   

\subsection{Opening a balanced hand with 11-12 HCP}

As first question we took is one which is accepted as common wisdom:
nowadays balanced hands with 11 or 12 HCP are opened much more frequent
in first and second hand.  We only consider the second hand if the
first hand does not fit the ``11-12 balanced'' profile; otherwise we
would  biase the outcome. 

\begin{table}
\caption{..}
\label{tbl:opening-11-12-bal-2}
\centering\footnotesize
\begin{tabular}{|c|r|r|r|r|}
\hline
\bf \ Decade \ & \bf \ Pass \ & \bf \ Opening \ & \bf \ Total \  & 
\bf \ \% Opened \ \\ \hline 
50-59 & 105 & 66 & 171 & 0.39 \\
60-69 & 44 & 44 & 88 & 0.50 \\
70-79 & 74 & 80 & 154 & 0.52 \\
80-89 & 47 & 61 & 108 & 0.56 \\
90-99 & 63 & 133 & 196 & 0.68 \\
00-09 & 363 & 803 & 1166 & 0.69 \\
10-13 & 190 & 354 & 544 & 0.65 \\ 
Total & 886 & 1541 & 2427 & \\
\hline
\end{tabular}
\end{table}

\tblref{opening-11-12-bal-2} shows the results. About 12.5\% (2427)  of the
records fitted the profile. The frequencies have
been aggregated in decades.   The percentages
suggest that a major change in style took place in the nineties, when the
frequency of opening 11-12 balanced hands went up from ``about half''
to ``about two-third''. After the nineties no major change appears to have
taken place.  The results also suggest that there was a marked
increase in frequency in the sixties.  
 
\subsection{Opening a hand with a  6-card and 0-9 HCP}

In the second analysis we selected from the hand records those in
which the dealer has some 6-card and less than 10 HCP. The reason we
only looked at the dealer position is because in other positions the
variation of the previous bids could easily biase the results (or
better: we lacked the time and energy to consider all the
consequences).  If the hand was opened we recorded the level, which
ranged from 1-4.  

\begin{table}
\caption{Absolute and relative frequencies of dealer actions with a
  6-card suit and less than 10 HCP. Data is aggregated per decade.}
\label{tbl:weak-two-1st}
\centering
\begin{tabular}{|c|r|r|r|r|r|r|r|r|r|r|r|}
\hline
\bf \ Decade \ & 
\multicolumn{2}{|c|}{\bf \ Pass \ }  &
\multicolumn{2}{|c|}{\bf \ 1 level \ } &
\multicolumn{2}{|c|}{\bf \ 2 level \ } &
\multicolumn{2}{|c|}{\bf \ 3 level \ } &
\multicolumn{2}{|c|}{\bf \ 4 level \ } &
\bf \ Total \  \\ 
\cline{2-11} &
\bf \ \ \# \ &\bf \ \ \% \ &
\bf \ \ \# \ &\bf \ \ \% \ &
\bf \ \ \# \ &\bf \ \ \% \ &
\bf \ \ \# \ &\bf \ \ \% \ &
\bf \ \ \# \ &\bf \ \ \% \ &
\\ \hline\hline
50-59& 89&0.86&2&0.02&10&0.10&3&0.03&0&0.00&104 \\ \hline
60-69& 57&0.84&2&0.03&5&0.07&4&0.06&0&0.00&68 \\ \hline
70-79& 69&0.86&0&0.00&8&0.10&3&0.04&0&0.00&80 \\ \hline
80-89& 32&0.64&6&0.12&8&0.16&4&0.08&0&0.00&50 \\ \hline
90-99& 78&0.70&4&0.04&19&0.17&11&0.10&0&0.00&112 \\ \hline
00-09&461&0.63&35&0.05&196&0.27&37&0.05&4&0.01&733 \\ \hline
10-13&198&0.58&14&0.04&87&0.26&37&0.11&4&0.01&340 \\ \hline
Total &984&&63&&333&&99&&8&&1487 \\ \hline
\end{tabular}
\end{table}

\tblref{weak-two-1st} lists the results, again aggregated per
decade.  About 7.5\% (1467) of the records match the profile. Since
the eighties there is a marked increase in the frequency in which these
hands are opened.  Unfortunately, the number of deals for the eighties
is very low. On manual inspection most hands there were opened at the
1-level turn out to be non-standard cases from the 1987 Bermuda
Bowl.  It is therefore more useful to look at the openings at the
higher levels.  

\insertfig{weak-two-1st}{\textwidth}{Cumulative frequencies of for a
  2/3/4-level opening with a  6 card and less than 10 HCP. Data is
  aggregated per decade.}

\figref{weak-two-1st} shows of the cumulative
frequencies for the seven decades.  Unlike the 11-12 balanced hands
the frequency keeps rising after the initial increase.  If we check
the data for the non-vulnerable situation\footnote{%
These data are not included in the paper; see the results directory at
the GitHub site} 
this trend is clear. In 2013 14 of these hands were opened at the
3-level, whereas only 6 at the 2-level.   

\subsection{Preempts}

Our third question is related to the previous one, but is more
genrally aimed at the frequency and nature of preempts. Defining the
notion of preempt precisely is not trivial. We consider a bid to be a
preempt if it satisfies the following conditions: (1) it is a jump
bid in a suit; (2) the bid is either an opening or an
overcall\footnote{%
Overcalls can be ``direct'' (2nd position, after an opening) or ``live (4th position,
after an opening and a response)}; 
(2) the bidder has less than 10 HCP; (4) the bid is made in the first
round of bidding.  This definition might be overly restrictive (for
example, a bid after an original pass is not considered a preempt),
but at least we know with high certainty that it is indeed a
preemptive bid. 

We also want to get information of the relationship between suit
length and the level at which the preempt is made. We decided to
compute for each preempt the L value , which we define as the length
of the longest  suit minus the level at which the bid is made.  Thus, a low value of L
indicates an aggressive preempt.  

\begin{table}
\caption{Frequency of preempts, aggregated per decade. The last column
lists the mean of the length of the longest suit minus the level at
which the preempt is made (the L value).}
\centering
\label{tbl:preempt}
\begin{tabular}{|c|r|r|r|c|c|}
\hline
\bf \ Decade \ & \bf \ No preempt \ & \bf Preempt \ & \bf Total  \ & 
\bf \% Preempt & \bf \ L \  \\ 
\hline\hline
50-59 & 1133 & 74 & 1,207 & 0.06 & 3.82 \\ \hline
60-69 & 794  & 38 & 832& 0.05 & 3.87 \\ \hline
70-79 & 971  & 52 & 1,023 & 0.05 & 3.79 \\ \hline
80-89 & 825 & 69 & 894 & 0.08 & 3.61 \\ \hline
90-99 & 1,556 & 143 & 1,699 & 0.08 & 3.71 \\ \hline
00-09 & 8,292 & 955 & 9,247 & 0.10 & 3.82 \\ \hline
10-13 & 4,225 & 533 & 4,758 & 0.11 & 3.50 \\ \hline
\hline
Total  & 17,796 & 1,864 & 19,660 &  & \\ 
\hline
\end{tabular}
\end{table}

\tblref{preempt} shows the results data for the Bermuda Bowl data
set. The frequency of preempts appears to have increased constantly
since the eighties. Over the course of six decades the frequency has
roughly speaking doubled. It should be pointed out that this query
might be biased somewhat due to the manual dealing of hands in older
tournaments. However, the increase is much larger than could be
explained by this bias (if there is such a bias at all, see
\tblref{balanced_first_hand}), 

It is difficult to draw conclusions from the mean L values. Given the
earlier results we would expect a decrease, but in particular the
data from the period 2000-2009 do not support this. 
We see an decrease in recent years, but more data is
needed to verify that this is not an accidental result. 
We should also point out that the L value overestimates the two-suited
hands.  For now we can only state that the length/level 
property of preempts needs to be studied in more depth in a follow-up.  

\subsection{Contested auction}


\begin{table}
\caption{..}
\centering
\begin{tabular}{|c|r|r|r|r|}
\hline
\bf \ Decade \ & \bf \ No preempt \ & \bf \ Preempt \ & \bf \ Total \  & 
\bf \ \% Preempt \ \\ \hline
\end{tabular}
\end{table}


\insertfig{contested-auction}{\textwidth-2cm}{Scatterplot and regression
line of the relation between year (X-axis) and frequency of a
contested auction (Y-axis). P-value of the linear model: 0.0421}

\section{Discussion}
\label{sec:discussion}

\paragraph{Acknowledgements}
This study would not have been possible without the work on a select
group of enthousiasts, who have started collecting hand records and
making these publicly available. In particular, the author wants to
acknowledge the support of Richard van Haaastrecht who provide through
his website\footnote{\url{http://www.bridgetoernooi.com/}} the data
for this study.  

\bibliographystyle{plain}
\bibliography{bridge}

\end{document}
